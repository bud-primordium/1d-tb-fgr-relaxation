\documentclass[12pt,a4paper]{article}
\usepackage[UTF8]{ctex}
\usepackage{amsmath,amssymb,amsfonts}
\usepackage{graphicx}
\usepackage{booktabs}
\usepackage{siunitx}
\usepackage{geometry}
\usepackage{float}
\usepackage{caption}
\usepackage{subcaption}
\usepackage{hyperref}
\usepackage{xcolor}
\usepackage{enumitem}

\geometry{left=2.5cm,right=2.5cm,top=2.5cm,bottom=2.5cm}

% 定义常用命令
\newcommand{\bra}[1]{\langle #1 |}
\newcommand{\ket}[1]{| #1 \rangle}
\newcommand{\braket}[2]{\langle #1 | #2 \rangle}
\newcommand{\mean}[1]{\langle #1 \rangle}
\newcommand{\dd}{\mathrm{d}}
\newcommand{\ii}{\mathrm{i}}

\title{\textbf{一维紧束缚模型中声子辅助带内弛豫的尺寸效应}}
\author{}
\date{}

\begin{document}

\maketitle

\begin{abstract}
    带内弛豫涉及多步声子辅助跃迁。当系统扩胞时,中间态数目增加,可能的弛豫路径(如 $1 \to 4$、$1 \to 2 \to 4$、$1 \to 2 \to 3 \to 4$ 等)也随之增多。本文基于一维紧束缚模型,采用 Fermi 黄金规则构建速率网络,探索带内弛豫的基本规律:弛豫速率如何随系统尺寸变化?哪些路径是主要的?哪些声子模式主导跃迁?

    数值结果表明:(1) 初态(带顶附近)的总跃迁速率 $\Gamma_{i_0}$ 满足 $\Gamma_{i_0} \propto N^\beta$,其中 $\beta \approx 0.03$,符合理论推导的 O(1) 预期,弛豫速率不随系统尺寸发散;(2) 低温下约 99\% 的弛豫轨迹仅需 3 步即可完成(多步路径占比 $f_{\rm aux} = P(n \geq 4) \approx 1\%$),平均跳数在不同 $N$ 下几乎不变(变异系数 $< 0.01$);(3) 跃迁集中于布里渊区边界附近的高频声子模式($q_{\rm peak} \approx \pi$),显著通道数远小于态数目。参数稳健性检验表明,上述结论在温度 $kT \in [0.01, 0.1]$ 与电声耦合 $\alpha \in [0.2, 1.0]$ 范围内稳定;能量展宽参数 $\sigma$ 需大于临界值 $\sigma_c \approx 0.07$ 以保证离散能级的准连续近似有效。
\end{abstract}

%=============================================================================
\section{引言}
%=============================================================================

\subsection{物理背景}

载流子带内弛豫是热载流子冷却、非辐射复合等过程的前驱步骤。在周期性体系中,高能电子态向低能态的弛豫需要通过声子发射(或吸收)完成能量转移。当能量差较大时,单次声子过程可能无法跨越整个能隙,需要多步跃迁依次进行。

当系统扩胞(如从原胞到超胞)时,布里渊区折叠导致能带离散点数增加,中间态数目 $\propto N$。弛豫路径的可能组合也随之增多——例如从态 1 到态 4,可以是直接跃迁 $1 \to 4$,也可以经过中间态 $1 \to 2 \to 4$、$1 \to 3 \to 4$、$1 \to 2 \to 3 \to 4$ 等。

\subsection{核心问题}

本文旨在探索带内弛豫的基本规律,具体包括:
\begin{enumerate}
    \item 弛豫速率是否随系统尺寸 $N$ 发散?
    \item 哪些弛豫路径是主要的?平均需要几步?
    \item 哪些声子模式主导跃迁?
\end{enumerate}

\subsection{研究背景}

在与施昊哲、谢昀城的前期讨论中,我们初步认为弛豫速率 $\Gamma$ 不随 $N$ 发散,但对于“强相干”与“退相干”条件下,两类辅助跃迁的物理图像存在困惑。经褚老师指正与进一步推导分析\footnote{详见施昊哲的\texttt{time\_compare.pdf}},认同室温下常见体系适合采用非相干速率网络框架处理。无量纲判据 $\mathcal{R}=\frac{\tau_{dwell}}{T_2} \gg 1$(停留时间与相干时间比值)表明相位记忆在一个跃迁周期内即丧失,因此可采用 Markov 速率网络描述。本文在此前提下进行探究。

\subsection{本文目标}

\begin{enumerate}
    \item 建立最小模型,定量验证弛豫速率的尺寸依赖性;
    \item 通过路径统计,确定主要的弛豫路径与平均步数;
    \item 分析声子模式分布,识别主导跃迁的声子动量;
    \item 检验结论对关键参数的稳定性。
\end{enumerate}

%=============================================================================
\section{理论框架}
%=============================================================================

本文采用无量纲约定:$\hbar = k_B = t_0 = 1$,其中 $t_0$ 为最近邻跳跃积分。

\subsection{Fermi 黄金规则}

考虑离散 Bloch 态 $\ket{k_i} \to \ket{k_j}$ 的声子辅助跃迁。根据 Fermi 黄金规则,单步跃迁速率为:
\begin{equation}
    W_{i \to j} = \frac{2\pi}{N} \sum_q |g_{ij}(q)|^2 \left[ (n_q + 1) \delta_\sigma(E_j - E_i + \omega_q) + n_q \delta_\sigma(E_j - E_i - \omega_q) \right]
    \label{eq:fgr}
\end{equation}
其中:
\begin{itemize}
    \item $\sum_q$ 形式上对所有声子模式求和,但动量守恒 $k_j = k_i + q \pmod{2\pi/a}$ 使得对给定 $(i,j)$ 只有唯一 $q_{ij} = k_j - k_i$ 满足选择定则,因此实际上每对 $(i,j)$ 仅对应一个有效声子模式;
    \item $g_{ij}(q) = \bra{\psi_i} \partial H / \partial Q_q \ket{\psi_j}$ 为电声耦合矩阵元\footnote{沿用上次报告的约定,将$1/N$因子提至外侧};
    \item $n_q = 1/[\exp(\omega_q / kT) - 1]$ 为玻色占据数;
    \item $\delta_\sigma$ 为高斯展宽函数,宽度 $\sigma$ 模拟有限寿命、无序等效应;
    \item 第一项对应声子发射($E_i > E_j$),第二项对应声子吸收($E_i < E_j$)。
\end{itemize}

\subsection{跃迁选择定则}

对于 Bloch 态之间的跃迁,存在以下约束:
\begin{enumerate}
    \item \textbf{动量守恒}:$k_j = k_i \pm q \pmod{2\pi/a}$;
    \item \textbf{能量守恒}:$|E_i - E_j| = \omega_q$(由 $\delta_\sigma$ 软化);
    \item \textbf{声子色散约束}:$\omega_q = \omega_{\max} |\sin(qa/2)|$。
\end{enumerate}
在本模型中,这些选择定则使速率矩阵 $W$ 呈现稀疏带状结构,详见附录\ref{sec:app_rate_matrix}.

\subsection{总跃迁速率的 O(1) 标度律}

定义态 $i$ 的总跃迁速率(逃逸速率)为:
\begin{equation}
    \Gamma_i = \sum_{j \neq i} W_{i \to j}
    \label{eq:gamma_def}
\end{equation}

在离散体系中,单个跃迁通道 $W_{i \to j}$ 带有显式的 $1/N$ 因子(式~\eqref{eq:fgr})。当 $N$ 增大时,能量窗口内满足选择定则的末态数目如何变化?

从连续极限的角度理解:在热力学极限下,对 $k$ 空间的求和可替换为积分
\begin{equation}
    \sum_k \to \frac{V}{(2\pi)^d} \int \dd^d k
\end{equation}
其中 $V \propto N$(一维情形 $V = Na$)。因此,可选末态数目 $\propto N$。结合Fermi 黄金规则:
\begin{equation}
    \Gamma_i \sim \underbrace{N}_{\text{可选末态数}} \times \underbrace{\frac{1}{N}}_{\text{单通道因子}} \times \underbrace{O(1)}_{\text{耦合权重}} \sim O(1)
    \label{eq:gamma_scaling}
\end{equation}

这解释了为何候选路径的组合增长不会导致弛豫速率发散:态密度的增长被 FGR 公式中的归一化因子抵消,总跃迁速率保持为常数量级。

\subsection{主方程}

速率网络的演化由主方程描述:
\begin{equation}
    \frac{\dd P_i}{\dd t} = \sum_j \left( W_{j \to i} P_j - W_{i \to j} P_i \right)
    \label{eq:master}
\end{equation}
矩阵形式为 $\dot{\mathbf{P}} = \mathbf{Q}^T \mathbf{P}$,其中生成矩阵 $\mathbf{Q}$ 定义为:
\begin{equation}
    Q_{ij} = \begin{cases}
        W_{i \to j},                       & i \neq j \\
        -\Gamma_i = -\sum_{j} W_{i \to j}, & i = j
    \end{cases}
\end{equation}

稳态分布 $\mathbf{P}_\infty$ 满足 $\mathbf{Q}^T \mathbf{P}_\infty = \mathbf{0}$ 且 $\sum_i P_{\infty,i} = 1$。数值求解采用 SVD 分解提取零空间,比特征值分解更稳健,见仓库\href{https://github.com/bud-primordium/1d-tb-fgr-relaxation/commit/e1ee6e1c5ec7de6929650d0c177e0e77ff0138fd}{\texttt{commit e1ee6e1}}.

\textbf{细致平衡}:速率矩阵近似满足 $W_{i \to j} / W_{j \to i} = \exp[-(E_j - E_i)/kT]$,使稳态趋近 Boltzmann 分布,在图\ref{fig:rate_matrix}中得到验证。

\subsection{Gillespie 算法}

动力学 Monte Carlo(KMC)采用 Gillespie 算法模拟单轨迹演化:
\begin{enumerate}
    \item 计算当前态 $i$ 的总出率 $\Gamma_i = \sum_j W_{i \to j}$;
    \item 抽样等待时间 $\Delta t = -\ln(r_1) / \Gamma_i$,其中 $r_1 \in (0,1)$ 为均匀随机数;
    \item 以概率 $p_{ij} = W_{i \to j} / \Gamma_i$ 抽样下一态 $j$。
\end{enumerate}
统计量包括跳数 $n_{\rm hop}$、总时间 $t_{\rm total}$、以及单步能量变化 $\Delta E = E_{\rm before} - E_{\rm after}$.

\subsection{弛豫时间定义}

本文采用两种互补的弛豫时间定义:
\begin{itemize}
    \item \textbf{ME 定义}:$\tau_{\rm ME}$ 为平均能量 $\mean{E}(t)$ 衰减到阈值的时间;
    \item \textbf{KMC 定义}:$\tau_{\rm KMC} = \mean{t_{\rm first\ passage}}$,即从初态出发首次到达终止阈值的平均时间。
\end{itemize}
两者绝对值不要求一致;本文关注它们随 $N$ 的变化趋势。

%=============================================================================
\section{计算模型}
%=============================================================================

\subsection{一维紧束缚模型}

考虑 $N$ 个格点的一维原子链,最近邻跳跃积分为 $t_0$。电子色散关系为:
\begin{equation}
    E(k) = 2t_0 \cos(ka)
\end{equation}
其中 $a$ 为晶格常数(取 $a = 1$)。

声子采用单原子链模型,色散关系为:
\begin{equation}
    \omega_q = \omega_{\max} \left| \sin\left( \frac{qa}{2} \right) \right|, \quad \omega_{\max} = \sqrt{\frac{2K}{M}}
\end{equation}
其中 $K$ 为弹簧常数,$M$ 为原子质量。

电声耦合采用 SSH 型调制:原子位移调制最近邻跳跃积分,耦合矩阵元为:
\begin{equation}
    g_{ij}(q) = \bra{\psi_i} \frac{\partial H}{\partial Q_q} \ket{\psi_j}
\end{equation}
在 Bloch 基底 $\ket{k}$ 下,该矩阵元具有解析形式:
\begin{equation}
    g(k_i \to k_f; q) \propto (e^{iqa} - 1)(e^{ik_i a} + e^{-ik_f a})
    \label{eq:g_bloch}
\end{equation}
由此可得:
\begin{enumerate}
    \item 第一项 $|e^{iqa} - 1|^2 = 4\sin^2(qa/2)$ 在 $q \to 0$ 时趋零(长波声子近似整体平移,对键长调制的耦合不起作用),在 $q = \pi/a$ 时取最大值,因此耦合\textbf{抑制小 $q$、增强 $q \approx \pi$};
    \item 第二项 $|e^{ik_i a} + e^{-ik_f a}|^2 = 2 + 2\cos[(k_i + k_f)a]$ 依赖初末态动量之和,产生\textbf{干涉}效应。
\end{enumerate}

\subsection{$\delta$ 函数展宽}

采用高斯展宽替代严格 $\delta$ 函数:
\begin{equation}
    \delta_\sigma(x) = \frac{1}{\sqrt{2\pi}\sigma} \exp\left( -\frac{x^2}{2\sigma^2} \right)
\end{equation}
展宽参数 $\sigma$ 模拟有限寿命、无序、热涨落等效应,控制离散谱变化到准连续极限。

\subsection{状态空间与初末态定义}

\textbf{状态空间}:本模型包含 \textbf{1 条电子能带}(一维最近邻紧束缚)和 \textbf{1 条声子色散支}(单原子链声学支)。对于系统尺寸 $N$,电子态空间由 $N$ 个离散 Bloch 态 $\{\ket{k_n}\}$ 构成,对应 $N$ 个离散波矢 $k_n = 2\pi n / (Na)$,$n = 0, 1, \ldots, N-1$。

\textbf{初态选取}:选取带顶附近最接近 $E_{\rm init} = E_{\min} + 0.9 (E_{\max} - E_{\min})$ 的态作为初态 $i_0$。

\textbf{终态判据}:弛豫过程在能量首次满足 $E \leq E_{\rm term}$ 时终止,其中 $E_{\rm term} = E_{\min} + 0.1 (E_{\max} - E_{\min})$(带底附近)。注意终态是一个\textbf{能量阈值区间},而非单个指定态。

\subsection{能量区间记号}

为便于描述弛豫路径,将能量归一化到 $[0,1]$ 并等分为 4 段,按从带顶到带底编号为 A--D。对于色散 $E(k) = 2t_0 \cos k$(不可约区 $k \in [0,\pi]$),这近似对应固定的 $k$ 区间:A: $[0, \pi/3]$,B: $[\pi/3, \pi/2]$,C: $[\pi/2, 2\pi/3]$,D: $[2\pi/3, \pi]$.

需要强调:A--D 是\textbf{能量段编号},不是固定的 $k$ 点编号,也不是“多条能带”。“$1 \to 2 \to 4$”等路径记号应理解为能量从高到低的跃迁序列。

图~\ref{fig:model} 展示模型示意图,表~\ref{tab:params} 展示计算参数。

\begin{figure}[htbp]
    \centering
    \includegraphics[width=0.9\textwidth]{../results/publication_figures/fig1_model_schematic.png}
    \caption{一维紧束缚模型示意图。左:单带色散 $E(k) = 2t_0 \cos k$(不可约区 $k \in [0,\pi]$)及能量区间 A--D;右:初态区间(带顶 10\%)与终态阈值区间(带底 10\%)。A--D 为能量段编号(非固定态编号/多能带)。}
    \label{fig:model}
\end{figure}

\begin{table}[htbp]
    \centering
    \caption{计算参数}
    \begin{tabular}{lll}
        \toprule
        参数      & 符号       & 值                    \\
        \midrule
        系统尺寸    & $N$      & 20, 40, 80, 160, 320 \\
        跳跃积分    & $t_0$    & 1.0                  \\
        弹簧常数    & $K$      & 1.0                  \\
        原子质量    & $M$      & 1.0                  \\
        电声耦合    & $\alpha$ & 0.5                  \\
        温度      & $kT$     & 0.025(低温)/ 0.5(高温对照) \\
        展宽      & $\sigma$ & 0.1                  \\
        KMC 轨迹数 & --       & 1000                 \\
        初态能量区   & --       & 带顶 10\%              \\
        终态判据    & --       & 带底 10\%              \\
        \bottomrule
    \end{tabular}
    \label{tab:params}
\end{table}



%=============================================================================
\section{结果}
%=============================================================================

回顾引言中提出的三个问题:$N$ 增大后弛豫会不会变慢甚至发散?新增中间态会不会让多步路径成为主流?需要哪类声子辅助跃迁?本节通过数值计算给出定量回答。

核心结论可概括为三条:(1) 总跃迁速率 $\Gamma_{i_0}$ 对 $N \geq 40$ 近似为常数量级,弛豫时间不随 $N$ 增长;(2) 多步路径占比 $f_{\rm aux} = P(n_{\rm hop} \geq 4)$ 不随 $N$ 增长,路径类型组成在 $N \geq 40$ 后进入稳定区间;(3) 跃迁集中于布里渊区边界附近的高频声子模式,$N \geq 40$ 时主导声子动量 $q_{\rm peak} \approx \pi$($N = 20$ 因有限尺寸效应偏离)。

\subsection{标度律验证}

图~\ref{fig:scaling} 展示总跃迁速率 $\Gamma$ 随系统尺寸 $N$ 的变化。采用幂律拟合 $\Gamma = A N^\beta$,排除 $N = 20$ 的有限尺寸效应后,得到 $\beta = 0.03$($R^2 = 0.74$)。

\begin{figure}[htbp]
    \centering
    \includegraphics[width=0.8\textwidth]{../results/publication_figures/fig2_scaling_laws.png}
    \caption{标度律验证。(a) 总跃迁速率 $\Gamma$ 随 $N$ 近似为常数,拟合指数 $\beta = 0.03$;(b) 平均能量步长收敛到有限值;(c) 弛豫时间在 $N \gtrsim 40$ 后稳定。}
    \label{fig:scaling}
\end{figure}

\begin{table}[htbp]
    \centering
    \caption{总跃迁速率数值结果}
    \begin{tabular}{cccccc}
        \toprule
        $N$      & 20   & 40   & 80   & 160  & 320  \\
        \midrule
        $\Gamma$ & 1.06 & 1.49 & 1.57 & 1.57 & 1.59 \\
        \bottomrule
    \end{tabular}
    \label{tab:gamma}
\end{table}

拟合指数 $|\beta| < 0.15$ 表明总跃迁速率不随系统尺寸显著变化。$N = 20$ 时 $\Gamma$ 偏低属于有限尺寸效应。这与式~\eqref{eq:gamma_scaling} 的理论预期一致。跃迁通道数目的定量分析见附录~\ref{sec:app_channels}。

\subsection{声子动量分布}

图~\ref{fig:qmode}(a) 是跃迁速率在 $(k_i, q)$ 空间的热图($k_i \in [0, 2\pi)$,$q \in [-\pi, \pi]$)。主要特征:

\begin{enumerate}
    \item \textbf{$q = 0$ 白线}:$q = 0$ 处 $|e^{iqa}-1|^2 = 0$,跃迁速率严格为零,形成一条水平白线;
    \item \textbf{四条斜向亮带,上下错开}:亮带斜率约为 $-1$,对应 $k_f = k_i + q \approx \text{const}$。$q > 0$ 与 $q < 0$ 区域的亮带在 $q = 0$ 附近并不对齐,而是错开约 $\pm 0.5$。这是因为整体速率由耦合强度、能量匹配、玻色因子共同决定,各因素的竞争使最亮位置偏离简单的 $k_f = \text{const}$ 线。
\end{enumerate}

图~\ref{fig:qmode}(b) 展示各 $q$ 模式的总跃迁速率 $\sum_i W_{i,i+q}$,峰值位于布里渊区边界($q \approx \pm\pi$)。图~\ref{fig:qmode}(c) 显示主导声子动量 $q_{\rm peak}$ 在 $N \geq 40$ 后收敛到 $\pi$。由式~\eqref{eq:g_bloch} 可知,$|e^{iqa} - 1|^2$ 在 $q = \pi$ 时最大,同时声子色散 $\omega_q \propto |\sin(qa/2)|$ 也在 $q \approx \pi$ 最大,因此\textbf{耦合强度与声子频率均在布里渊区边界达到峰值}。

\begin{figure}[htbp]
    \centering
    \includegraphics[width=0.98\textwidth]{../results/publication_figures/fig9_q_distribution.png}
    \caption{声子动量分布。(a) 跃迁速率在 $(k_i, q)$ 空间的热图:$q = 0$ 处因 $|e^{iqa}-1|^2 = 0$ 形成突兀白线;四条斜向亮带(斜率 $-1$)在 $q = 0$ 附近上下错开;(b) 各 $q$ 模式的总跃迁速率,峰值在 $q \approx \pm\pi$;(c) 主导声子动量 $q_{\rm peak}$ 在 $N \geq 40$ 后收敛到 $\pi$。}
    \label{fig:qmode}
\end{figure}

\subsection{路径统计}

为定量刻画弛豫路径的复杂度,定义以下统计量:
\begin{itemize}
    \item \textbf{跳数} $n_{\rm hop}$:从初态到满足终止条件的跃迁次数;
    \item \textbf{多步路径占比} $f_{\rm aux} := P(n_{\rm hop} \geq 4)$:需要 4 步或更多才能完成弛豫的轨迹占比;
    \item \textbf{吸收占比} $r_{\rm abs}$:所有跃迁中声子吸收($\Delta E < 0$)的比例。
\end{itemize}

图~\ref{fig:path} 展示路径统计结果,对比低温($kT = 0.025$)与高温($kT = 0.5$)两种情况:

\begin{figure}[htbp]
    \centering
    \includegraphics[width=0.85\textwidth]{../results/publication_figures/fig3_path_statistics.png}
    \caption{路径统计(低温/高温对比)。(a) 跳数分布的关键概率:$P(n=3)$ 与 $f_{\rm aux} = P(n \geq 4)$ 随 $N$ 的变化;(b) 平均跳数 $\mean{n_{\rm hop}}$;(c) 能量步长分布;(d) 吸收占比 $r_{\rm abs}$ 随 $N$ 的变化。}
    \label{fig:path}
\end{figure}
\begin{itemize}
    \item 低温下 $P(n = 3) \approx 99\%$,$f_{\rm aux} \approx 1\%$,$\mean{n_{\rm hop}} \approx 3.0$;
    \item 高温下步数分布变宽,$\mean{n_{\rm hop}} \approx 3.9$,$f_{\rm aux} \approx 33\%$;
    \item $f_{\rm aux}$ 与 $r_{\rm abs}$ 均\textbf{不随 $N$ 增长}, $N$ 增大带来更多中间态,但它们并未被有效利用。
\end{itemize}

\subsection{路径类型分析}

仅用跳数统计还无法回答“哪类中间态路径更显著”(例如 A$\to$B$\to$D 与 A$\to$C$\to$D 哪个更常见),因此我们进一步对每条轨迹做粗粒化的路径类型统计:将轨迹映射为能量区间序列(A--D),并压缩连续重复得到路径类型。

图~\ref{fig:motif} 展示不同 $N$ 下路径类型的组成(低温与高温对照)。结果有两个要点:

\textbf{(i) 低温下},路径几乎完全由 A$\to$B$\to$C$\to$D 与 A$\to$B$\to$D 两类构成,A$\to$C$\to$D 仅为千分量级。从 $N = 20$ 的有限尺寸点到 $N \geq 40$ 后,A$\to$B$\to$D 的占比提高并进入相对稳定区间(约 0.36--0.49),未呈现随 $N$ 继续增长的趋势。

\textbf{(ii) 高温下},由于声子吸收引入回跳,出现一定比例的“绕行/回跳”路径(归入 Other,约 15\%),同时 A$\to$C$\to$D 的占比上升到百分量级,但同样未随 $N$ 增大系统性漂移。

总体而言,扩胞确实引入更多候选中间态,但在本模型与参数范围内,\textbf{有效的路径结构没有随 $N$ 走向更复杂}。代表性轨迹与速率矩阵结构的详细分析见附录~\ref{sec:app_traj} 和~\ref{sec:app_rate_matrix}。

\begin{figure}[htbp]
    \centering
    \includegraphics[width=0.8\textwidth]{../results/publication_figures/fig10_path_motifs.png}
    \caption{路径类型随系统尺寸的变化(能量区间粗粒化;A--D 为能量段编号而非具体态编号)。将能量归一化并等分为 4 段,按从带顶到带底编号为 A--D;每条轨迹映射为能量区间序列并压缩连续重复得到路径类型。图中展示 A$\to$D、A$\to$B$\to$D、A$\to$C$\to$D、A$\to$B$\to$C$\to$D 四类代表路径,其余归入 Other(含回跳等非单调序列)。}
    \label{fig:motif}
\end{figure}

%=============================================================================
\section{讨论}
%=============================================================================

\subsection{候选路径与有效路径}

扩胞引入更多离散中间态,可写出的路径组合数快速增长。然而数值结果表明,\textbf{有效路径并未随之爆炸}。原因在于选择定则的约束:能量守恒、动量守恒与声子色散共同限制了每一步的可能目标。速率矩阵呈稀疏带状结构(附录~\ref{sec:app_rate_matrix}),新增中间态大多“存在但不被访问”。

从通道分析(附录~\ref{sec:app_channels})看,前 3--7 个目标态即覆盖 90\% 以上的跃迁概率,这与路径统计中多步路径占比不随 $N$ 增长的结论相吻合。

\subsection{局限性}

\begin{itemize}
    \item \textbf{模型简化}:简谐声子、单电子图像、一维最近邻紧束缚;
    \item \textbf{唯象参数}:展宽 $\sigma$ 需满足准连续条件($\sigma \gtrsim 0.1$);
    \item \textbf{路径分辨率有限}:能量区间统计是粗粒化的,不区分具体的 $k$ 点;
    \item \textbf{三维外推}:三维材料需额外考虑多支色散与多声子过程。
\end{itemize}

%=============================================================================
\section{结论}
%=============================================================================

本文通过一维紧束缚模型的数值计算,探索了带内弛豫的基本规律。数值结果表明:

\begin{enumerate}
    \item \textbf{弛豫速率}:初态的总跃迁速率 $\Gamma_{i_0}(N) \propto N^\beta$, $\beta = 0.03 \pm 0.01$,不随系统尺寸发散;
    \item \textbf{多步路径占比}:$f_{\rm aux}(N) = P(n_{\rm hop} \geq 4)$ 不随 $N$ 增长(低温约 1\%,高温约 33\%),平均跳数 $\mean{n_{\rm hop}}$ 保持稳定;
    \item \textbf{主导声子动量}:$N \geq 40$ 时 $q_{\rm peak} \approx \pi$,对应布里渊区边界的高频声子模式。
\end{enumerate}
\textbf{参数稳健性}(详见附录~\ref{sec:app_sensitivity}):上述结论在 $kT \in [0.01, 0.1]$、$\alpha \in [0.2, 1.0]$ 范围内稳定;但要求展宽参数 $\sigma \gtrsim 0.1$ 以保证离散能级的准连续近似有效。ME 与 KMC 两种方法的一致性验证见附录~\ref{sec:app_consistency}。

%=============================================================================
\newpage
\appendix
\section{附录:计算流程与代码结构}
%=============================================================================

\subsection{计算依赖树}

\begin{verbatim}
solve_master_equation(W, P0, t_span)           [master_equation.py]
|-- W = build_rate_matrix(N, params)           [fermi_golden_rule.py]
|   |-- k_grid = 2*pi*n/(Na), n = 0,1,...,N-1
|   |-- E_grid = 2*t0*cos(ka)                  [电子色散]
|   |-- for each (i,j):
|   |   |-- dE = E_i - E_j
|   |   |-- q = k_j - k_i (mod 2*pi/a)         [动量守恒,代码通过 j=(i+q) 构造]
|   |   |-- w_q = w_max*|sin(qa/2)|            [声子色散]
|   |   |-- delta_sigma(dE - w_q) or (dE + w_q) [能量守恒]
|   |   |-- |g(k_i -> k_j; q)|^2
|   |   +-- W_ij = (2*pi/N)*|g|^2*delta * [n_B+1 or n_B]
|   +-- return W, k_grid, E_grid
|-- P_inf = stationary_distribution(W)         [SVD 零空间]
|-- <E>(t) = sum_i E_i * P_i(t)
+-- tau_ME = first t s.t. <E(t)> <= E_eq + theta*(<E(0)>-E_eq)

run_trajectory(W, E, i0)                       [kinetic_monte_carlo.py]
|-- Gillespie algorithm:
|   |-- Gamma_i = sum_j W_ij                   [总逃逸速率]
|   |-- dt = -ln(r1) / Gamma_i                 [等待时间]
|   +-- j = sample(W_ij / Gamma_i)             [跃迁目标]
+-- 记录 {n_hops, step_sizes, t_total}
\end{verbatim}

\subsection{关键公式与代码对应}

\begin{table}[H]
    \centering
    \small
    \begin{tabular}{p{3.5cm}p{5cm}p{5cm}}
        \toprule
        \textbf{物理量}    & \textbf{公式/定义}                                    & \textbf{代码函数}                                            \\
        \midrule
        电子色散            & $E(k) = 2t_0 \cos(ka)$                            & \texttt{tb\_electron\_1band.dispersion}                  \\[0.3em]
        声子色散            & $\omega_q = \omega_{\max}|\sin(qa/2)|$            & \texttt{phonon\_1atom.dispersion\_monatomic}             \\[0.3em]
        展宽 $\delta$     & 高斯展宽                                              & \texttt{fermi\_golden\_rule.delta\_broadened}            \\[0.3em]
        FGR 速率矩阵        & $W_{ij} = W_{i \to j}$                            & \texttt{fermi\_golden\_rule.build\_rate\_matrix}         \\[0.3em]
        主方程生成矩阵         & $Q_{ii} = -\sum_j W_{i \to j}$                    & \texttt{master\_equation.generator\_from\_rates}         \\[0.3em]
        稳态分布            & $\mathbf{Q}^T \mathbf{P} = 0$, $\sum P = 1$       & \texttt{master\_equation.stationary\_distribution}       \\[0.3em]
        $\tau_{\rm ME}$ & 能量衰减到阈值                                           & \texttt{master\_equation.relaxation\_time\_from\_energy} \\[0.3em]
        Gillespie 单步    & 抽样 $(dt, \text{next})$                            & \texttt{kinetic\_monte\_carlo.gillespie\_step}           \\[0.3em]
        显著通道数           & $D_{\rm eff} = \#\{j \mid p_{ij} > \varepsilon\}$ & \texttt{channel\_analysis.effective\_out\_degree}        \\[0.3em]
        参与数             & $D_{\rm pr} = 1/\sum p_{ij}^2$                    & \texttt{channel\_analysis.participation\_ratio}          \\
        \bottomrule
    \end{tabular}
\end{table}

\subsection{数值实现要点}

\begin{enumerate}
    \item \textbf{稳态分布求解}:采用 SVD 分解 $\mathbf{Q}^T$ 的零空间,避免直接求逆的数值不稳定性;
    \item \textbf{初态/终态一致性}:ME 与 KMC 使用相同的初态选取与终止阈值;
    \item \textbf{统计收敛}:KMC 默认 1000 条独立轨迹,误差棒使用标准差。
\end{enumerate}

\subsection{复现脚本}

\begin{table}[H]
    \centering
    \begin{tabular}{ll}
        \toprule
        \textbf{脚本}                             & \textbf{功能}                        \\
        \midrule
        \texttt{run\_scaling\_experiment.py}    & 主实验($N$ 扫描 + 路径统计)                 \\
        \texttt{run\_sensitivity.py}            & 参数稳健性检验($\sigma$/$kT$/$\alpha$ 扫描) \\
        \texttt{generate\_core\_figures.py}     & 核心图表                               \\
        \texttt{generate\_appendix\_figures.py} & 附录图表                               \\
        \bottomrule
    \end{tabular}
\end{table}

%=============================================================================
\section{附录:补充分析}
\label{sec:appendix_supp}
%=============================================================================

\subsection{跃迁通道数目分析}
\label{sec:app_channels}

为量化速率矩阵的稀疏程度,定义以下指标:
\begin{itemize}
    \item \textbf{跃迁概率}:$p_{ij} = W_{i \to j} / \Gamma_i$,即归一化的跃迁速率;
    \item \textbf{显著通道数}:$D_{\rm eff}(i) = \#\{j \mid p_{ij} > 0.01\}$,即从态 $i$ 出发、跃迁概率超过 1\% 的目标态个数;
    \item \textbf{参与数}:$D_{\rm pr}(i) = 1 / \sum_j p_{ij}^2$,类似 IPR(inverse participation ratio),用于度量跃迁概率的分散程度。若跃迁均匀分布到 $M$ 个态,则 $D_{\rm pr} = M$;若集中在单一通道,则 $D_{\rm pr} = 1$。
\end{itemize}

图~\ref{fig:channels} 展示通道数目分析。虽然 $D_{\rm eff}$ 和 $D_{\rm pr}$ 随 $N$ 有所增大,但远小于“全连通”
情形。若无选择定则约束,每个态可跃迁到其它所有 $N-1$ 个态。累积概率曲线显示,前 3--7 个通道即可覆盖 90\% 以上的跃迁概率,说明跃迁高度集中于少数目标态。

\begin{figure}[htbp]
    \centering
    \includegraphics[width=0.95\textwidth]{../results/publication_figures/fig8_effective_channels.png}
    \caption{跃迁通道数目分析。(a) 显著通道数 $D_{\rm eff}$ 随归一化能量的分布,不同颜色对应不同 $N$;能带边缘的态通道数较多,带中部较少。(b) 平均显著通道数 $\mean{D_{\rm eff}}$(蓝)与参与数 $\mean{D_{\rm pr}}$(橙)随 $N$ 的变化,灰色虚线为全连通预期 $N-1$;两者均远小于 $N$,表明速率矩阵是稀疏的。(c) 累积概率曲线:将各态的跃迁目标按概率从大到小排列,高能态约 3 个通道即达 90\%,低能态约需 7 个。}
    \label{fig:channels}
\end{figure}

\subsection{代表性轨迹}
\label{sec:app_traj}

图~\ref{fig:traj} 展示 KMC 代表性轨迹。能量随时间呈阶梯式下降(“跨栏式”弛豫),低温下以声子发射为主,不同 $N$ 的轨迹形态相近。

\begin{figure}[htbp]
    \centering
    \includegraphics[width=0.8\textwidth]{../results/publication_figures/fig4_trajectories.png}
    \caption{KMC 代表性轨迹。低温下以声子发射为主,不同 $N$ 的轨迹形态相近。}
    \label{fig:traj}
\end{figure}

\subsection{速率矩阵结构}
\label{sec:app_rate_matrix}

图~\ref{fig:rate_matrix}(a,b) 展示速率矩阵 $W_{ij}$ 的热图($\log_{10}$ 色标,$N=60$)。主要特征如下:

\begin{enumerate}
    \item \textbf{主对角线空白}:$i = j$ 处 $W_{ii} = 0$(无自跃迁),且 $q \to 0$ 时耦合项 $|e^{iqa}-1|^2 \to 0$,导致对角线及其附近严格为零或很弱;
    \item \textbf{反对角线方向的两条亮带}:最显著的特征是沿左下到右上方向的两条亮带,位于 $i+j \approx 50$ 与 $i+j \approx 70$。由式~\eqref{eq:g_bloch} 的干涉项 $|e^{ik_i a}+e^{-ik_f a}|^2 = 2+2\cos[(k_i+k_f)a]$,当 $k_i + k_f = 2\pi$ 时取最大值,对应 $i+j = N = 60$(因为 $k = 2\pi \cdot \text{index}/N$)。但整体速率还需乘以能量匹配因子与 $|e^{iqa}-1|^2$ 权重,因此最亮位置被推移到 $i+j = 60$ 附近的两条偏移脊线($\approx 50$ 与 $\approx 70$);
    \item \textbf{低温下一支不完整}:低温时 $n_q \ll 1$,吸收项整体被压制,导致其中一条亮带出现缺段;高温时吸收增强,缺段被补齐,两条亮带均完整可见,图形趋于对称。这与图~\ref{fig:rate_matrix}(c) 中吸收占比随温度的变化一致。
\end{enumerate}

图~\ref{fig:rate_matrix}(d) 通过细致平衡验证数值实现的正确性:高温下 $\ln(W_{ij}/W_{ji})$ 对 $(E_j - E_i)/kT$ 的斜率接近理论值 $-1$。

\begin{figure}[htbp]
    \centering
    \includegraphics[width=0.95\textwidth]{../results/publication_figures/fig5_rate_matrix.png}
    \caption{速率矩阵分析。(a,b) 低温与高温的速率矩阵 $\log_{10}(W_{ij})$,呈带状结构;(c) 吸收占比随能量的变化;(d) 细致平衡验证。}
    \label{fig:rate_matrix}
\end{figure}

\subsection{参数稳健性检验}
\label{sec:app_sensitivity}

前述结果基于 baseline 参数($\sigma = 0.1$,$kT = 0.025$,$\alpha = 0.5$)。本节通过单因子扫描检验结论的稳健性。

图~\ref{fig:sensitivity} 展示标度指数 $\beta$ 对三个参数的依赖。

\begin{figure}[htbp]
    \centering
    \includegraphics[width=0.95\textwidth]{../results/publication_figures/fig6_sensitivity.png}
    \caption{参数稳健性检验:标度指数 $\beta$ 对三个参数的依赖。(a) 展宽 $\sigma$:$\sigma < 0.07$ 时 O(1) 失效;(b) 温度 $kT$:在 $[0.01, 0.1]$ 范围内稳定;(c) 电声耦合 $\alpha$:在 $[0.2, 1.0]$ 范围内稳定。}
    \label{fig:sensitivity}
\end{figure}

\begin{table}[htbp]
    \centering
    \caption{展宽参数对标度指数的影响}
    \begin{tabular}{ccc}
        \toprule
        $\sigma$      & $\beta$       & 状态                    \\
        \midrule
        0.05          & 0.51          & 失效($|\beta| > 0.15$)  \\
        0.07          & 0.17          & 临界                    \\
        \textbf{0.10} & \textbf{0.03} & \textbf{baseline(稳定)} \\
        0.15          & 0.005         & 稳定                    \\
        0.20          & 0.005         & 稳定                    \\
        \bottomrule
    \end{tabular}
    \label{tab:sigma}
\end{table}

$\sigma$ 是控制离散能级“准连续化”程度的关键参数。当 $\sigma$ 过小时($< 0.07$),能量匹配更苛刻,导致 $\beta$ 偏大;$\sigma = 0.07$ 时虽然 $\beta$ 已降至 $0.17$ 左右,但仍略超过 $|\beta| < 0.15$ 判据。稳妥起见,选取 $\sigma \gtrsim 0.1$(baseline 取值)以保证 O(1) 结论可靠。

在 $kT \in [0.01, 0.1]$ 与 $\alpha \in [0.2, 1.0]$ 范围内,$\beta \approx 0.03 \pm 0.01$,结论完全稳定。

\subsection{ME 与 KMC 一致性验证}
\label{sec:app_consistency}

图~\ref{fig:consistency} 展示两种方法的一致性验证。

\begin{figure}[htbp]
    \centering
    \includegraphics[width=0.8\textwidth]{../results/publication_figures/fig7_consistency.png}
    \caption{ME 与 KMC 一致性。(a) 归一化弛豫时间 $\tau/\tau(N=80)$ 随 $N$ 的变化;(b) $\tau_{\rm ME}$ 与 $\tau_{\rm KMC}$ 的条形图对比;(c) 稳态分布与 Boltzmann 分布的 KL 散度。}
    \label{fig:consistency}
\end{figure}

$\tau_{\rm ME}$ 与 $\tau_{\rm KMC}$ 绝对值相差约 5--7 倍,这源于定义差异:$\tau_{\rm ME}$ 采用阈值 $\theta = 0.01$(能量衰减到初始偏移的 1\%),若假设指数衰减 $\exp(-t/\tau)$,则 $\tau_{\rm ME}/\tau \approx -\ln(0.01) \approx 4.6$;而 $\tau_{\rm KMC}$ 为首达时间,两者物理含义不同。

尽管绝对值不同,归一化后的曲线显示两者在 $N \geq 40$ 后快速收敛到常数量级,\textbf{均不发散}。稳态分布与 Boltzmann 分布的 KL 散度在全部测试尺寸下均保持 $10^{-2}$ 量级(展宽与截断效应导致的系统偏差),说明稳态分布与理论预期基本一致。

\end{document}
